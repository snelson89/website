\documentclass[11pt]{article}
\usepackage[margin=1in]{geometry}
\usepackage{fontawesome}
\usepackage{xcolor}
\usepackage{hyperref}
\hypersetup{
  colorlinks=true,
  urlcolor=teal
}
\usepackage{multicol}

\setlength{\parindent}{0ex}
\setlength{\parskip}{1ex}


\begin{document}
\pagenumbering{gobble}

\noindent\begin{minipage}{.6\linewidth}
	\subsection*{LIN 350 -- Experimental Phonetics}
\end{minipage}
\begin{minipage}{.4\linewidth}
	\begin{flushright}\subsection*{Summer 2023}\end{flushright} 
\end{minipage}

\noindent\rule{\linewidth}{1pt}
\vspace{.2cm}

\noindent\begin{minipage}[t]{.5\linewidth}
	\noindent Instructor: Scott Nelson\\
	Office: SBS N235 \\
	%\faPhone \hspace{1pt} +1 (517) 884 4328 \\
	\faEnvelopeO \hspace{1pt} \href{mailto:scott.nelson@stonybrook.edu}{scott.nelson@stonybrook.edu}
\end{minipage}%
\begin{minipage}[t]{.5\linewidth}
	\begin{flushright}
		Class Time: T/Th 9:30am -- 12:55pm \\
		Class Location: SBS N310 \\
		Office Hours: by appointment \\
%		Lecture: T/Th 12:40 -- 2:00\\
%		103 Erickson Hall\\
%%		\rule{.5\linewidth}{.5pt}\\
%		Recitation : F 9:10 -- 10:00 (Section 001)\\
%		F 10:20 -- 11:10 (Section 002) \\
%		F 11:30 -- 12:20 (Section 003) \\
%		218 A Berkey Hall
	\end{flushright}
\end{minipage}
\vspace{.5cm}
\subsubsection*{Office Hours/Contact}

I will be holding both in person and virtual office hours via Zoom by appointment. 
If you would like to talk with me outside of class, please do not hesitate to send me an e-mail so we can work out a time to meet up. 
You may access my ``virtual office'' using the meeting link provided here: \href{https://stonybrook.zoom.us/j/4091340647}{https://stonybrook.zoom.us/j/4091340647}. 
\textit{Note: Zoom does not require you to download their software to join a meeting.}
\vspace{.25cm}

If you are contacting me via e-mail, please put [LIN 350] in the subject line. I will respond within two hours from 9am-5pm on Monday-Friday. All other times I will respond sporadically, but at least within half a day.  

You may also communicate with me via our class Slack channel (details below). In theory I should be able to respond to Slack posts quicker. Please limit this type of communication to quick or general questions. Any communication about class performance or grades needs to be done via e-mail.

\section*{Materials and Resources}

\textbf{Brightspace} - The official course Brightspace page will be used to host readings and post grades. A supplemental course website will be used for everything else.

\textbf{Class website} - A website for the class can be found at \href{https://www.scott-nelson.net/350SU23.html}{this link}. Additional class material will be posted here including a course log and links to Zoom/Google Forms/etc.... Please check this page regularly as you are responsible for all information posted there for this class.

\textbf{Slack} - You can join the class Slack channel \href{https://join.slack.com/t/lin350summer2023/shared_invite/zt-1wlogdm2b-JtDb~jPe89ky1jLfmbZeUg}{here}. Slack is a messaging app designed to be used for workplace communication. You can find an introduction on how to use it \href{https://slack.global.ssl.fastly.net/0cc2/pdfs/users_guide.pdf?fbclid=IwAR0IRj8DYxytM76m_j44wgg9-OWUmFMKfxLXfCM2Iv7qVyWP-5ZBo_T7Gsw}{here}. It is also good for class communication as it allows for easier and less formal communication between myself and the students. This is good for quick questions or off topic thoughts you want to share with me or other students. It also allows for sharing files/pictures/etc... so it can be used for all different types of communication. I recommend downloading the app for either your phone or desktop.

\textbf{Textbook} - There is no required textbook for this class. Readings will come from a variety of sources and will be posted on the class Brightspace page.



\section*{Course Description}

Introduction to common experimental methods for studying the sounds used in human language. Topics include basic speech acoustics, acoustic analysis, oral and nasal airflow, static palatography, linguography and electroglottography, as well as design of perception experiments. Students will learn the physical processes affecting each experimental variable and common methods of analyzing each kind of data. Students will get hands-on experience with each analysis method and will use two or more types of data to explore a hypothesis about sound structure in English or some other language of interest. Students will learn how to use software for making measurements and analyzing data. Students will learn to assess the validity of claims about language based on their understanding of the scientific method as applied to speech. The course will give students a solid foundation for further courses in laboratory skills relevant to assessment of normal and disordered speech and for pursuing research, either as undergraduate researchers, or in the early stages of graduate work.

\section*{Grading/Structure of Course}

Every class meeting will be broken up into three sections: {\color{red}Lecture}, {\color{blue}Praat}, and \textbf{Lab}. 
The {\color{blue}Praat} and \textbf{Lab} portions will be interactive and will require students to bring a laptop that is able to run Praat to class. 
If you do not have access to a laptop, please consult with the \href{https://it.stonybrook.edu/services/student-laptop-loaner-program}{Laptop Loaner Program}.

\begin{center}
\begin{tabular}{l l l}
	\hline
	A. & Final Project & 40\% \\
	B. & Reflection Essay & 10\% \\
	C. & Lab Reports (8) & 40\% \\
	D. & General Participation & 10\%  \\
	\hline
\end{tabular}
\end{center}

\subsection*{A. Final Project}

Each student will be required to do an acoustic analysis on a topic of their choosing. 
The analysis must involve a comparison of some type between two speech communities. 
This can include speakers of different languages, speakers of the same language but different dialects, or disordered and non-disordered speakers.
The main goal of this project is for students to apply the technical skills learned in this class to new data.
A secondary goal is to provide practice for students presenting their work to others.
There are three parts to the final project: a short proposal, an in-class presentation, and a written report.

You must submit a project topic proposal to me for approval by \textbf{Friday, August 4 at 5pm}. 
The proposal can be a short paragraph outlining the analysis you plan to perform. The goal of the proposal is to make sure you have a plan in place. 
I may request changes to your plan if necessary. 
Each student will also be required to give an approximately 10-15 minute presentation on their topic. 
After each presentation, there will be a 3-5 minute question period where students will ask the presenter further questions about their project. 
Presentations will take place on \textbf{August 15} and \textbf{August 17}. 
The final report should be 3-5 pages (single spaced, 1in margin, 12pt font) and is due \textbf{Friday, August 18 at 5pm}.

\subsection*{B. Reflection Essay}

Alongside the final draft of your project report you will turn in a reflection essay. This will be a one page document (single spaced, 1in margin, 12pt font) where you reflect on your final project and the class as a whole. It should minimally include what you consider to be the strengths and weaknesses of your final project and what you may have done differently if you had more time. Other than that, the topic of your reflection essay is open ended. Some possible directions to take it include: what you liked and didn't like about the course, how you will use what you learned in this course moving forward, or general suggestions for the structure of this course in the future.

\subsection*{C. Lab Reports}

In weeks 2-5, each class period will end with a lab.
These labs are designed to give students hands on experience measuring and analyzing acoustic data. 
Each lab will be completed individually and involves writing up a short lab report.
The lab report is due by the start of the class period one week after the assignment of the lab (i.e. - if the lab is on a Tuesday, it is due by the start of the following Tuesday's class).

Every lab report should contain the following four sections: \textit{Introduction}, \textit{Material and methods}, \textit{Results}, \textit{Discussion and conclusion}. 
The introduction provides a brief overview of what is being measured and why it is of interest. 
The materials and methods sections should give an explicit description of how things were measured.
This includes both descriptions of materials used (software, microphone) as well as techniques used (e.g. - measured the vowel at its midpoint).
Someone should be able to replicate your process exactly after reading this section.
The results section contains the results.
This should be a combination of both prose and figures.
The discussion and conclusion section summarizes all of the results and discusses interesting and significant findings.

\subsection*{D. General Participation}

Students are expected to be active participants in class.
If you are unable to make a class, please try to let me know ahead of time.


\section*{Student Accessibility Support Center Statement}

If you have a physical, psychological, medical, or learning disability that may impact your course work, please contact the Student Accessibility Support Center, Stony Brook Union Suite 107, (631) 632-6748, or at \href{mailto:sasc@stonybrook.edu}{sasc@stonybrook.edu}. They will determine with you what accommodations are necessary and appropriate. All information and documentation is confidential.

%Students who require assistance during emergency evacuation are encouraged to discuss their needs with their professors and the Student Accessibility Support Center. For procedures and information go to the following website: \href{https://ehs.stonybrook.edu/programs/fire-safety/emergency-evacuation/evacuation-guide-people-physical-disabilities}{https://ehs.stonybrook.edu/programs/fire-safety/emergency-evacuation/evacuation-guide-people-physical-disabilities} and search Fire Safety and Evacuation and Disabilities.

\section*{Academic Integrity Statement}

Each student must pursue his or her academic goals honestly and be personally accountable for all submitted work. Representing another person's work as your own is always wrong. Faculty is required to report any suspected instances of academic dishonesty to the Academic Judiciary. Faculty in the Health Sciences Center (School of Health Professions, Nursing, Social Welfare, Dental Medicine) and School of Medicine are required to follow their school-specific procedures. For more comprehensive information on academic integrity, including categories of academic dishonesty please refer to the academic judiciary website at\\ \href{https://www.stonybrook.edu/commcms/academic_integrity/index.html}{https://www.stonybrook.edu/commcms/academic\_integrity/index.html}.

\section*{Critical Incident Management}

Stony Brook University expects students to respect the rights, privileges, and property of other people. Faculty are required to report to the Office of Student Conduct and Community Standards any disruptive behavior that interrupts their ability to teach, compromises the safety of the learning environment, or inhibits students' ability to learn. Faculty in the HSC Schools and the School of Medicine are required to follow their school-specific procedures. Further information about most academic matters can be found in the Undergraduate Bulletin, the Undergraduate Class Schedule, and the Faculty-Employee Handbook.


\end{document}